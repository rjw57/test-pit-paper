\documentclass[conference]{IEEEtran}
\usepackage{amsmath}
\usepackage{microtype}

\title{A Probabilistic Approach to Test Pitting}
\author{%
    \IEEEauthorblockN{Rich Wareham}
    \IEEEauthorblockA{Signal Processing and Communications Laboratory\\
    Department of Engineering, University of Cambridge\\
    Email: rjw57@cam.ac.uk}
}

\begin{document}

\maketitle

\begin{abstract}
    We develop a simple threshold-based test for determining the significance of
    one test pit when compared to others within some shared context, for example
    a single village. We do this by developing a statistical model for test pit
    evolution and then use the model to provide a data-driven test for
    significance. For our purposes, `significant' is taken to mean `surprising
    given the wider context'. Such significant test pits can be used to localise
    settlement or other processes leading to larger-scale potter deposition.
\end{abstract}

\section{Introduction}

In this text we consider a probabilistic approach to analysing test pit data in
which we assign a probability to each test pit in relation to its wider context
within a site or, in a mathematically equivalent manner, the probability of a
given site given the wider context of the area. We do this by comparing weights
and/or sherd counts which are partitioned into different eras.

Conceptually we wish to find some formula which reflects our prior belief that a
test pit which has a larger proportion of, for example, Roman pottery in it in
comparison to other test pits in the area is in some way more `surprising',
i.e.\ has a lower probability, than test pits whose pottery proportions match
the site as a whole.

Throughout this document we shall consider a single test pit within a collection
of pits on the same site and use the weight of pottery as the measured value.
The mathematics works for both pottery weight or sherd counts and may trivially
be extended to consider the probability of entire sites within wider contexts of
settlement by considering an individual site as a single large test pit. Indeed,
such an approach is likely, given the greater quantities of data, to lead to
firmer conclusions at the expense of coarser localisation.

For the purposes of illustration, we will consider potter to fall into the
categories Roman, Early Medieval, Late Medieval, High Medieval. This
categorisation is purely conventional and the technique below will work with any
pottery categorisation assuming that no one sherd may fall into more than one
category and that all shreds being considered have an associated category.

\section{Notation}

Assume that there are $N$ test pits in the site. For the $i$-th test pit we
denote the weight of Roman era pottery as $R_i$, the weight of Early Medieval
pottery as $E_i$, Late Medieval as $L_i$ and High Medieval as $H_i$. The total
weight of pottery is denoted as $T_i$ and we make the assumption that all
pottery in the pit comes from one of our designated eras. That is to say that we
assume
\[
    T_i = R_i + E_i + L_i + H_i.
\]

Given these individual test pit weights we can derive the \emph{total} weights.
For example the total weight of Roman pottery, $R$, is given by
\[
    R = R_1 + R_2 + \cdots + R_N \equiv \sum_{i=1}^N R_i
\]
and similar expressions can be written down for the total amounts of Early, Late
and High Medieval, $E$, $L$ and $H$ respectively and also the total weight of
pottery over all pits, $T = R + E + L + H$.

Sometimes, for convenience's sake, we refer to the \emph{normalised} weights,
for example $R'_i \equiv R_i / T_i$ which can be interpreted as the
\emph{proportion} of, in this instance, Roman pottery within the pit.

\section{Assumptions and Definitions}

Given a particular pit we assume that there is some notional ``true'' proportion
of pottery which we cannot measure directly. Instead we can only measure the
proportion which survives. For example if 400g of Roman pottery went into the
pit, we may only get 4g which survives. This loss of pottery is statistical;
even if we know that, on average, 1\% of Roman pottery survives that does not
mean that finding 4g of Roman pottery in a pit means exactly 400g of pottery
went into it. Instead we can say that our best estimate is that there was
originally 400g of pottery plus or minus some statistical error. For the
purposes of this discussion we shall assume that pottery loss is approximated by
a Poisson process. Such a process is one in which a number of individual
independent events contribute to an overall effect. In this case we assume that
each event which causes some sherd of pottery to decay and disappear is
independent of each other event. We further assume this variable to be
approximately normally distributed in the range in which we are analysing. That
is to say that we are dealing with levels of pottery in which the Poisson
distribution is reasonably approximated by the Gaussian distribution. \emph{De
facto} this will be valid when the total sherd counts get above six or seven.

Having justified the use of normal distributions, our observation model is to
assume that the unmeasurable ``true'' proportion of Roman pottery which went
into a test pit is $r'_i$ and is distributed according to
\[
    r'_i \sim \mathcal{N}\left(\mu = \frac{R_i}{T_i}, \sigma = \frac{1}{\sqrt{R_i T_i}} \right).
\]
This reflects our prior belief that the actual proportion of Roman pottery, when
compared to pottery from other eras, put \emph{into} the pit is most likely to
be the proportion we subsequently find \emph{within} the pit allowing for some
error. This error is smaller if we find more Roman pottery and also if we find
more pottery in general. Similar equations can be formed for $e_i, \ell_i$ and
$h_i$.

This model has been somewhat `plucked from the air' in that it is proposed as a
reasonable model of our prior belief whose behaviour matches one's intuition as
an arch\ae ologist. Other models are possible.

\section{Analysis}

Let us consider the entire site as a single test pit. Using our observation
model above we expect the true proportion of Roman pottery over the entire site,
$r'$, which we cannot directly measure, to have the following likelihood:
\[
    p(r' | R, T) = \frac{\sqrt{R T}}{\sqrt{2\pi}} \exp \left(
        - 2 \left(r' - \frac{R}{T}\right)^2 R T  \right).
\]
This is simply the result of substituting our expected mean and variance into
the standard form for a Gaussian PDF.

If we let $R_{\setminus i}$ and $T_{\setminus i}$ be the Roman era and total
weights \emph{excluding} test pit $i$ then we note that $p(r' | R_{\setminus
i},T_{\setminus i})$ is necessarily independent of $p(r' | R_i, T_i)$ and hence
\[
    p(r' | R_{\setminus i}, T_{\setminus i}, R_i, T_i) = p(r' | R_{\setminus i},
    T_{\setminus i}) p(r' | R_i, T_i).
\]

Eliding some trivial Bayesian analysis and marginalising out $r'$ we note that
the posterior probabilities are symmetric:
\begin{align*}
    P(R_{\setminus i},T_{\setminus i} | R_i, T_i) &= P(R_i, T_i | R_{\setminus
    i},T_{\setminus i}) \\
    & = \int p(x | R_{\setminus i}, T_{\setminus i}) p(x | R_i, T_i) \, \mathrm{d}x.
\end{align*}
We have also integrated out the true unmeasurable value $x$ leaving us with an
expression only in terms we can measure.  This expression is the integral of a
product of two Gaussian PDFs.  The product of two Gaussian PDFs in $x$ is itself
a Gaussian PDF in $x$ scaled by another Gaussian PDF of the parameters. Since
that second PDF is independent of $x$ it becomes a simple scaling term for the
integral. The first PDF integrates to unity, by definition, and thus we are left
with a single Gaussian PDF in the parameters:
\begin{align*}
    P(R_{\setminus i},T_{\setminus i} | R_i, T_i) &= P(R_i, T_i | R_{\setminus
    i},T_{\setminus i}) \\ &=
    \frac{1}{\sqrt{2 \pi B^{(i)}_R}} \, \exp \left(
    - \frac{{A^{(i)}_R}^2}{2 B^{(i)}_R}
    \right)
\end{align*}
where
\[
    A^{(i)}_R = \frac{R_i}{T_i} - \frac{R_{\setminus i}}{T_{\setminus i}}, \quad
    B^{(i)}_R = \frac{1}{R_i T_i} + \frac{1}{R_{\setminus i} T_{\setminus i}}.
\]

If we define the \emph{surprise} of test pit $i$ to be the reciprocal of the
posterior probability then
\[
    S^{(R)}_i = \sqrt{2\pi B^{(i)}_R} \, \exp \left( \frac{{A^{(i)}_R}^2}{2B^{(i)}_R} \right)
\]
where $S^{(R)}_i$ is the surprise of the proportion of Roman pottery in test pit
$i$. Replacing $R$ with $E, L$ and $H$ we obtain similar results for Early, Late
and High Medieval pottery proportions.

We may further interpret the value within the exponentiation as the square of
the number of standard deviations the observed pottery count is from that
expected by the null hypothesis. We may therefore compute a $p$-value via the
error function or simply choose a threshold; a 3-sigma threshold would, for
example, correspond to:

\[
    \left|\frac{A^{(i)}_R}{\sqrt{2 B^{(i)}_R}}\right| \ge 3.
\]

\section{Conclusion}

In the section above we developed the statistical model of test pit evolution
over time and obtained a statistic which may be used to evaluate the
significance of one test pit in comparison to others within a similar location.
Specifically, for the example of Roman pottery, the value
\[
    n^{(R)}_i = \frac{R_i \cdot T^{-1}_i - R_{\setminus i} \cdot T^{-1}_{\setminus
    i}}{\sqrt{2}\sqrt{(R_i T_i)^{-1} + (R_{\setminus i}T_{\setminus i})^{-1}}}
\]
may be computed for the $i$-th test pit and can be interpreted as a number of
standard deviations away from the null hypothesis. Consequently, a value of
$|n^{(R)}_i| \ge 3$ for test pit $i$ can be used to conclude that the proportion
of Roman pottery in test pit $i$ differs from neighbouring test pits in a
statistically significant way. The sign of $n^{(R)}_i$ indicates whether this
differences is in the form of a surprisingly little (-ve) or a surprisingly
large (+ve) amount of pottery. Similar expressions can be formed for the other
eras.

It should be noted that this statistic is merely a value which can be used to
directly compare test pits based on the assumptions stated above. Specifically,
we have assumed that test pits which contain a relatively large amount of
pottery from one era or a relatively large amount of pottery in general are more
indicative of the ``true'' pottery distribution and that such indicative test
pits which vary greatly from the rest of the site are significant. The aim of
the statistic is to flag those test pits worthy of closer inspection.

\end{document}

% vim:sw=4:sts=4:et:tw=80
