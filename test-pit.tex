\documentclass[a4paper]{paper}

\title{A Probabilistic Approach to Test Pitting}
\author{Rich Wareham}

\begin{document}

\maketitle

\section{Introduction}

In this text we consider a probabilistic approach to test pit data in which we
assign a probability to each test pit in relation to its wider context within a
site or, in a mathematically equivalent manner, the probability of a given site
given the wider context of the area. We do this by comparing weights or sherd
counts which are partitioned into different eras.

Conceptually we wish to find some formula which reflects our prior belief that a
test pit which has a larger proportion of, for example, Roman pottery in it in
comparison to other test pits in the area is in some way more 'surprising' (has
a lower probability) than test pits whose pottery proportions match the site as
a whole.

Throughout this document we shall consider a single test pit within a collection
of pits on the same site and use the weight of pottery as the measured value.
The mathematics works for both pottery weight or sherd counts and may trivially
be extended to consider the probability of sites within wider contexts by
considering an individual site as a single large test pit.

\section{Notation}

Assume that there are $N$ test pits in the site. For the $i$-th test pit we
denote the weight of Roman era pottery as $R_i$, the weight of Early Medieval
pottery as $E_i$, Late Medieval as $L_i$ and High Medieval as $H_i$. The total
weight of pottery is denoted as $T_i$ and we make the assumption that all potter
in the pit comes from one of our designated eras. That is to say that we assume
\[
    T_i = R_i + E_i + L_i + H_i.
\]

Given these individual test pit weights we can derive the \emph{total} weights.
For example the total weight of Roman pottery, $R$, is given by
\[
    R = R_1 + R_2 + \cdots + R_N \equiv \sum_{i=1}^N R_i
\]
and similar expressions can be written down for the total amounts of Early, Late
and High Medieval, $E$, $L$ and $H$ respectively and also the total weight of
pottery over all pits, $T = R + E + L + H$.

Sometimes, for convenience's sake, we refer to the \emph{normalised} weights,
for example $R'_i \equiv R_i / T_i$ which can be interpreted as the
\emph{proportion} of, in this instance, Roman pottery within the pit.

\section{Assumptions and Definitions}

Given a particular pit we assume that there is some notional ``true'' proportion
of pottery which we cannot measure directly. Instead we can only measure the
proportion which survives. For example if 400g of Roman pottery went into the
pit, we may only get 4g which survives. This loss of pottery is statistical;
even if we know that, on average, 1\% of Roman pottery survives that does not
mean that finding 4g of Roman pottery in a pit means exactly 400g of pottery
went into it. Instead we can say that our best estimate is that there was
originally 400g of pottery plus or minus some statistical error. For the
purposes of this discussion we shall assume that pottery loss is approximated by
a Poisson process.

In practical terms this means that given a weight of pottery in a pit of $R_i$
we assume that this number is a random variable with variance equal to $R_i$. We
further assume this number to be normally distributed. There is admittedly
little to justify this assumption beyond convenience of mathematics and that is
matches our prior intuition that test pits with a great deal of pottery in are
likely to have proportions closer to the ``true'' values than test pits with
comparatively little pottery in.

Our observation model is therefore to assume that the proportion of Roman
pottery which could survive in test pit $i$, $r'_i$, is distributed according to
\[
    r'_i \sim \mathcal{N}\left(\frac{R_i}{T_i}, \frac{1}{\sqrt{R_i T_i}} \right).
\]
Again this reflects our prior belief; the actual proportion of Roman pottery put
into the pit is most likely to be the proportion we find within the pit and the
error in this value goes down the more Roman pottery we find and the more
pottery which we find in general. Similar equations can be found for $e_i,
\ell_i$ and $h_i$.

\section{Analysis}

Let us consider the entire site as a single test pit. Using our observation
model above we expect the proportion of Roman pottery to have the following
likelihood:
\[
    p(x | R, T) = \frac{\sqrt{R T}}{\sqrt{2\pi}} \exp \left(
        - 2 \left(x - \frac{R}{T}\right)^2 R T  \right).
\]
This is simply the result of substituting our expected mean and variance into
the standard form for a Gaussian PDF.

If we let $R$ and $T$ be the Roman era and total weights \emph{excluding} test
pit $i$ then we note that $p(x | R,T)$ is necessarily independent of $p(x | R_i,
T_i)$ and hence
\[
    p(x | R, T, R_i, T_i) = p(x | R, T) p(x | R_i, T_i).
\]

Eliding some trivial Bayesian analysis and marginalising out $x$ we note that
the posterior probabilities are symmetric:
\[
    P(R,T | R_i, T_i) = P(R_i, T_i | R,T)
    = \int p(x | R, T) p(x | R_i, T_i) \, \mathrm{d}x.
\]

Pleasingly this is the integral of a product of two Gaussian PDFs. The product
of two Gaussian PDFs in $x$ is itself a Gaussian PDF in $x$ scaled by another
Gaussian PDF of the parameters. Since that second PDF is independent of $x$ it
becomes a simple scaling term for the integral. The first PDF integrates to
unity by definition and thus we are left with a single Gaussian PDF in the
parameters:
\[
    P(R,T | R_i, T_i) = P(R_i, T_i | R,T) =
    \frac{1}{\sqrt{2 \pi B_R}} \, \exp \left(
    - \frac{A_R^2}{2 B_R}
    \right)
\]
where
\[
    A_R = \frac{R_i}{T_i} - \frac{R}{T}, \quad
    B_R = \frac{1}{R_i T_i} + \frac{1}{R T}.
\]

If we define the \emph{surprise} of test pit $i$ to be the reciprocal of the
posterior probability then
\[
    S^{(R)}_i = \sqrt{2\pi B_R} \, \exp \left( \frac{A_R^2}{2B_R} \right)
\]
where $S^{(R)}_i$ is the surprise of the proportion of Roman pottery in test pit
$i$. Replacing $R$ with $E, L$ and $H$ we obtain similar results for Early, Late
and High Medieval pottery proportions.

\end{document}

% vim:sw=4:sts=4:et:tw=80
